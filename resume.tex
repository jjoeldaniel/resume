\documentclass[letterpaper,11pt]{article}

\usepackage{latexsym}
\usepackage[empty]{fullpage}
\usepackage{titlesec}
\usepackage{marvosym}
\usepackage[usenames,dvipsnames]{color}
\usepackage{verbatim}
\usepackage{enumitem}
\usepackage[hidelinks]{hyperref}
\usepackage{fancyhdr}
\usepackage[english]{babel}
\usepackage{tabularx}
\input{glyphtounicode}


%----------FONT OPTIONS----------
% sans-serif
% \usepackage[sfdefault]{FiraSans}
% \usepackage[sfdefault]{roboto}
% \usepackage[sfdefault]{noto-sans}
% \usepackage[default]{sourcesanspro}

% serif
% \usepackage{CormorantGaramond}
% \usepackage{charter}


\pagestyle{fancy}
\fancyhf{} % clear all header and footer fields
\fancyfoot{}
\renewcommand{\headrulewidth}{0pt}
\renewcommand{\footrulewidth}{0pt}

% Adjust margins
\addtolength{\oddsidemargin}{-0.5in}
\addtolength{\evensidemargin}{-0.5in}
\addtolength{\textwidth}{1in}
\addtolength{\topmargin}{-.5in}
\addtolength{\textheight}{1.0in}

\urlstyle{same}

\raggedbottom
\raggedright
\setlength{\tabcolsep}{0in}

% Sections formatting
\titleformat{\section}{
  \vspace{-4pt}\scshape\raggedright\large
}{}{0em}{}[\color{black}\titlerule \vspace{-5pt}]

% Ensure that generate pdf is machine readable/ATS parsable
\pdfgentounicode=1

%-------------------------
% Custom commands
\newcommand{\resumeItem}[1]{
  \item\small{
    {#1 \vspace{-2pt}}
  }
}

\newcommand{\resumeSubheading}[4]{
  \vspace{-2pt}\item
    \begin{tabular*}{0.97\textwidth}[t]{l@{\extracolsep{\fill}}r}
      \textbf{#1} & #2 \\
      \textit{\small#3} & \textit{\small #4} \\
    \end{tabular*}\vspace{-7pt}
}

\newcommand{\resumeSubSubheading}[2]{
    \item
    \begin{tabular*}{0.97\textwidth}{l@{\extracolsep{\fill}}r}
      \textit{\small#1} & \textit{\small #2} \\
    \end{tabular*}\vspace{-7pt}
}

\newcommand{\resumeProjectHeading}[2]{
    \item
    \begin{tabular*}{0.97\textwidth}{l@{\extracolsep{\fill}}r}
      \small#1 & #2 \\
    \end{tabular*}\vspace{-7pt}
}

\newcommand{\resumeSubItem}[1]{\resumeItem{#1}\vspace{-4pt}}

\renewcommand\labelitemii{$\vcenter{\hbox{\tiny$\bullet$}}$}

\newcommand{\resumeSubHeadingListStart}{\begin{itemize}[leftmargin=0.15in, label={}]}
\newcommand{\resumeSubHeadingListEnd}{\end{itemize}}
\newcommand{\resumeItemListStart}{\begin{itemize}}
\newcommand{\resumeItemListEnd}{\end{itemize}\vspace{-5pt}}

%-------------------------------------------
%%%%%%  RESUME STARTS HERE  %%%%%%%%%%%%%%%%%%%%%%%%%%%%


\begin{document}

%----------HEADING----------
\begin{center}
    \textbf{\Huge \scshape Joel Daniel Rico} \\ \vspace{1pt}
    \small \href{mailto:joeldanielrico@csu.fullerton.edu.com}{\underline{joeldanielrico@csu.fullerton.edu}} $|$ 
    \href{https://linkedin.com/in/joeldanielrico}{\underline{linkedin.com/in/joeldanielrico}} $|$
    \href{https://github.com/jjoeldaniel}{\underline{github.com/jjoeldaniel }}
\end{center}


%-----------CAREER OBJECTIVE-----------
\section{Career Objective}
  Aspiring Software Engineer with a passion for problem-solving and solution-based programming. Strong
  work ethic with proficiency in communication in team-based environments.



%-----------EDUCATION-----------
\section{Education}
  \resumeSubHeadingListStart
    \resumeSubheading
      {California State University, Fullerton}{Fullerton, CA}
      {Bachelor of Science in Computer Science}{Aug. 2023 -- Dec. 2025}
  \resumeSubHeadingListEnd


%-----------PROJECTS-----------
\section{Projects}
    \resumeSubHeadingListStart
      \resumeProjectHeading
          {\textbf{JPass} $|$ \emph{Python, Flask, Postgres, JavaScript, Vercel}}{Feb. 2023 -- Present}
          \resumeItemListStart
            \resumeItem{Developed a password manager for securely storing user passwords with \textbf{bcrypt}, a password-hashing function}
            \resumeItem{Implemented login and signup flow with OpenID and OAuth support for Google and Email authentication}
            \resumeItem{Created accessible, mobile-friendly web user interface}
            \resumeItem{Built with Python and Flask framework, with a relational database for storing encrypted user passwords}
          \resumeItemListEnd
      \resumeProjectHeading
          {\textbf{keyphrases.rs} $|$ \emph{Rust}}{March 2023 -- Present}
          \resumeItemListStart
            \resumeItem{Developed an open-source library for efficient and accurate extraction of keywords from large volumes of text}
            \resumeItem{Implemented Rapid Automatic Keyword Extraction (RAKE) algorithm}
            \resumeItem{Designed the library to be flexible, customizable, and scalable, allowing users to adjust the library's parameters and configurations to fit their needs}
            \resumeItem{Optimized the library's performance and memory usage to ensure efficient processing of large volumes of text data}
            \resumeItem{Contributed to the open-source community by sharing the library's source code and documentation on GitHub and crates.io}
          \resumeItemListEnd
      \resumeProjectHeading
          {\textbf{Trends Analysis} $|$ \emph{Python, KeyBERT}}{Jan. 2022 -- Present}
          \resumeItemListStart
            \resumeItem{Conducted data analysis on a Discord server with over 1,000 members to 
identify trends and patterns in user behavior}
            \resumeItem{Identified key trends and patterns in user activity, including peak hours of usage, most popular channels, and frequently used words and phrases}
            \resumeItem{Utilized a Bidirectional Encoder Representations from Transformers (BERT) model library to extract relevant keywords from Discord messages}
            \resumeItem{Extracted relevant data using Discord API and web scraping techniques to collect information on user activity, message content, and channel usage}
            \resumeItem{Visualized data using charts, graphs, and other visualization tools to highlight trends and patterns in user behavior}
          \resumeItemListEnd
      \resumeProjectHeading
          {\textbf{Genius API Wrapper} $|$ \emph{Python, Requests, Web Scraping}}{Dec. 2022 -- Jan. 2023}
          \resumeItemListStart
            \resumeItem{Developed a Python-based Genius API wrapper to facilitate access to the Genius lyrics database, enabling users to search for and retrieve lyrics and other song-related data}
            \resumeItem{Published on Python Package Index (PyPI), resulting in over 5,000 downloads and numerous positive reviews and feedback from users}
            \resumeItem{Utilized Requests HTTP library to communicate with Genius REST API}
            \resumeItem{Tested and validated accuracy and effectiveness in retrieving relevant and up-to-date song data, including lyrics, album art, artist information, and song metadata}
            \resumeItem{Implemented the API wrapper to provide a range of functionality, including searching for song lyrics, retrieving song metadata, and accessing song annotations and explanations}
          \resumeItemListEnd
    \resumeSubHeadingListEnd


%-----------PROGRAMMING SKILLS-----------
\section{Technical Skills}
 \begin{itemize}[leftmargin=0.15in, label={}]
    \small{\item{
     \textbf{Languages}{: Rust, Python, Java, C++, JavaScript, HTML, CSS, SQL (Postgres, SQLite) } \\
     \textbf{Tools}{: Git, Docker, Flask, FastAPI, {\fontfamily{lmr}\selectfont\LaTeX}} \\
    }}
 \end{itemize}


 %-----------EXTRACURRICULAR-----------
\section{Extracurricular}
    \resumeSubHeadingListStart
      \resumeProjectHeading
          {\textbf{Association For Computing Machinery (ACM) } $|$ \emph{Member, Contributor}}{Aug. 2022 -- Present}
          \resumeItemListStart
            \resumeItem{Co-organized a hackathon event, attracting over 150 applicants and 9 sponsors}
            \resumeItem{Collaborated with other club members to write open source software, contributing to the development of software tools and applications that benefited the community}
          \resumeItemListEnd
    \resumeSubHeadingListEnd


%-------------------------------------------
\end{document}
